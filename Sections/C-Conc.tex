\resetdatestamp

\chapter{Conclusion and future directions}\label{sec:ch5}

\newpage

In this dissertation, we conducted direct numerical simulation (DNS) studies of the effect of turbulence on accelerating the growth of cloud droplets in the early stage of shallow cloud development, particularly inside the adiabatic cloud core. Numerous sets of numerical experiments were performed to study the impact of turbulence on droplet geometric collisions, droplet hydrodynamic interaction, and the interactions of collision and condensation. 

In past studies, the turbulence has been shown to exert a positive impact on droplet growth and accelerates the formation of warm rain. It has been widely accepted that turbulence enhances droplet collision statiscs through the effects of clustering of droplets, increasing droplet relative motion, and modifying the response of the droplet to the local disturbance flow. The three corresponding collision statistics are the droplet radial distribution function, the radial relative velocity, and the collision efficiency. The effects of turbulence on these three statistics are not the same. Specifically, our results show that the turbulence enhancement through the clustering effect and the relative motion effect is weaker than the enhancement of collision efficiency. Because of the importance of the collision efficiency effect, an accurate parameterization is required. Parameterizations such as \citet{Riemer2005}, \citet{Xue2008}, and \citet{Franklin2008} on the droplet collision kernel based on the lookup table of gravitational collision efficiencies are inaccurate. Additionally, the parameterizations based on DNS studies such as \citet{Franklin2007} and \citet{Ayala2008b} included the Reynolds number as an important parameter. However, in Chapter \ref{sec:ch2} it was demonstrated that the Reynolds number from DNS, which is calculated based on the computational domain size and thus is entirely artificial, has a secondary effect on the droplet collision statistics relative to the local dissipation rate and the inertial response of the droplet to the local flow (i.e., the Stokes number). The physical reason for our finding is that the scale of turbulence affecting droplet collisions is on the order of the mean separation distance between cloud droplets, which is of the order of the Kolmogorov length scale in cumulus clouds and far below our smallest computational domain size. It follows that the computational $R_\lambda$ becomes irrelevant and the turbulence can be characterized by the dissipation rate alone. A new parameterization for the turbulent geometric collision kernel that excludes the Reynolds number is therefore propsed. 

To model the effect of turbulence on the evolution of the droplet size distribution (DSD), droplet condensational growth must be included in addition to the collisional growth. One of the most intriguing results in Chapter \ref{sec:ch4} is that in the case of pure condensation, the effect of turbulence is small. However, the interplay between condensation and collision in broadening the DSD in turbulence is found to be substantial. The main reason is that condensation growth leads to a narrow DSD with more similar-sized droplets and the turbulence enhancement of collisions is particularly strong for similar-sized drop pairs.

Finally we remark on the limitation of this study and some suggestions for future work. It has been found that the evolution of the DSD and the rain formation time highly depend on the initial shape of the DSD and the droplet number concentration. Therefore, simulations of different initial DSDs are to be conducted to better understand its dependency. In addition, the initial DSD used in this study is taken from flight observations which represent an average over a long sampling time and a wide sampling volume. In this case, the initial DSD is not guaranteed to be representative of the steady-state DSD from aerosol activation and condensational growth in adiabatic cloud cores. However, with the continuous advancement of in-situ and laboratory measurement technology such as HOLO-DEC \citep{GRL2017} and PI chamber \citep{Desai2018}, representative sampling of the DSD near the cloud base inside adiabatic cores may be possible in the near future. It is also desirable to include the aerosol activation process to enable cloud particles to grow from the very beginning (i.e., dry aerosols in sub-cloud regions). Besides, the model can also be modified to study other microphysics processes such as ice nucleation which is poorly parameterized for deep convective clouds and cirrus clouds, and particle electrification which is potentially important in aerosol scavenging and droplet collisions.  

\cleardoublepage 
