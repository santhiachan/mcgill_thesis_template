\resetdatestamp

\chapter{Introduction}

%\epigraph{A change in the weather is sufficient to recreate the world and ourselves.}{\emph{Marcel Proust} (1871 -- 1922)}

\section{Background}



\subsection{Clouds, weather, and climate}
	
Shallow convective clouds, such as marine stratocumulus clouds and trade-wind cumuli, represent an important component of the weather and climate systems of the Earth. In the first place, they substantially influence the Earth's radiation budget by reflecting the incoming short-wave radiation. This cooling of the atmosphere affects the large-scale dynamics as a response to the decreased radiative forcing. For example, the albedo of shallow marine clouds in the tropical and subtropical regions is about 0.3-0.8 \citep{ Chen2012, Painemal2012}, more than fivefold larger than the albedo of the ocean surface \citep [$\approx$ 0.06-0.1,] []{Payne1972}. However, by virtue of their shallowness and closeness to the ocean surface, these clouds radiate at nearly the same intensity as the surface and do not contribute to modulating the outgoing longwave radiation \citep{Stevens2009}. Consequently, shallow clouds have a strong net radiative effect, which in turn greatly impacts the intensity and distribution of storm-tracks in the mid-latitudes \citep{Wang2014} and the rain belts in the Inter Tropical Convergence Zone (ITCZ) \citep{Voigt2016}.

Secondly, clouds have a determinant influence on the Earth's hydrological cycle by redistributing the water on the planet through the interaction with atmospheric circulation and the processes of precipitation. Rain produced from warm clouds accounts for 31\% of the total rainfall on the planet and 72\% of the total rain area in tropical regions \citep{Lau2003}. The warm-rain processes can sometimes even result in natural hazards. For instance, the 2013 Great Colorado Flood along the eastern slopes of the Colorado Rocky Mountains was mainly caused by an unusually severe warm-rain process. During the event, a large concentration of small-sized raindrops and a copious amount of precipitable water were observed \citep{Friedrich2016}. 18 counties impacted by the floods were declared by the Federal Emergency Management Agency (FEMA) as federal disaster areas and damages exceeding \$2 billion in U.S. dollars were produced \citep{Gochis2013}. 

\subsection{Warm-rain initiation and condensation-collision size gap}

It is widely accepted that the growth process of a droplet in warm clouds consists of three basic sub-processes \citep[e.g., ][]{Rogers1989} : 1) droplets form from the activation of cloud condensation nuclei (CCN); 2) droplets grow by diffusion of water vapor (condensation); 3) droplets grow by collision-coalescence. For condensation, the approximate growth rate of a droplet in radius is inversely proportional to its size but directly proportional to the supersaturation in its immediate environment when solute and curvature effects are neglected, viz.
\begin{equation}
\frac{dr}{dt}\propto \frac{S}{r},
\end{equation}
where $S$ is the supersaturation and $r$ the droplet radius. The implication is that small droplets grow faster than large droplets. As a consequence, two droplets starting with different sizes would grow to a similar size in the end. It is known that the process of condensational growth dominates until a droplet reaches around 15 $\mu m$ in radius \citep{Pruppacher1997}. For larger droplets, the collision process takes over to form raindrops. In still air collision occurs when heavier droplets capture smaller ones lying in their paths under the influence of gravity. To initiate the collisional process effectively, a broad droplet size distribution (i.e., the co-existence of big drops and small droplets) is required. In general, droplets need to grow to at least 30-40 $\mu m$ for the gravitational collisional process to be effective \citep{Mason1975, Grabowski2013}. The underlying reason is that before this stage, the relative fall speed between droplets is very small and the collision efficiency becomes negligible. Specifically, when a droplet moves relative to the flow, a local flow field around the droplet surface called the disturbance flow is generated. The disturbance flow causes two approaching droplets to have a tendency to move away from each other and avoid collision. In particular, small droplets have weak inertia and tend to follow streamlines and thus have a low probability of collision, or a low collision efficiency. As a consequence, a condensation-collision size gap ($r \approx 15-30 \mu m$) exists where both condensational growth and collisional growth are inefficient.

However, radar observations show that warm rain can form within 15-20 minutes \citep{Szumowski1997, Knight2002, Goke2007}. and heavy precipitation \citep{Stevens2016} and wide droplet size distributions (DSDs) \citep{Warner1969, GRL2017,Siebert2017, Witte2017} have frequently been observed in shallow convective systems. Nevertheless, classical condensational growth theory for an air parcel rising adiabatically from the cloud base fails to reproduce the observed features in a reasonable timescale due to the growth size gap \citep{Jonas1996}.

\subsection{Possible mechanisms to broaden the droplet size spectrum}

The conflicting results between observation and theory pose a challenge for the cloud physics community for decades. In the literature, several mechanisms have been proposed to explain how warm-rain can be initiated efficiently. They include small-scale turbulence, aerosol effects, and cloud-scale mixing. Interested readers may refer to the review articles by \citet{Vaillancourt2000}, \citet{Devenish2012}, and \citet{Grabowski2013} for a comprehensive overview of these mechanisms. Here, a brief summary is given below.

\subsubsection{Small-scale turbulence}

Turbulence, though difficult to define precisely, is characterized by its randomness, intermittency, multi-scale interactions, and features of dissipation, diffusion and mixing \citep{TL1972}. Turbulence is ubiquitous in nature. Cumulus clouds are turbulent. Laboratory experiments \citep[e.g., ][]{Bordas2013} and numerical modeling studies \citep[e.g., ][]{Wang2000, Wang2008} have yielded evidence that turbulence may enhance cloud droplet collision-coalescence growth when the droplets are still small. The large eddy simulation (LES) experiments performed by \citet{Wyszogrodzki2013} show that turbulence can shorten the initiation time of warm rain and increase the amount of rainfall. 

So far, three major effects of turbulence have been found crucial to enhance the probability for collision or the collision kernel \citep{Wang2009}: 
(1) Turbulence redistributes droplets in space and thus creates droplet clustering (i.e., the turbulence clustering effect which affects the radial distribution function); (2) turbulence increases the droplet relative motion and modifies the droplet settling velocity (i.e., the transport effect); (3) turbulence modifies the droplet hydrodynamic response to the local disturbance flow which is induced by the flow passing by the droplet surface. This is so-called the turbulence hydrodynamic effect. The overall enhancement of turbulence through the above three mechanisms depends on the intensity of turbulence and the size of the droplets. 

\subsubsection{Aerosol effects}

Aerosols, including natural -- sulfates, sea salt, organic aerosols, etc. -- and anthropogenic -- soots, cloud seeding materials, etc. -- are the most common cloud condensation nuclei. Arguably, the physical and chemical properties of aerosols not only affect the cloud macroscopic properties such as the optical property \citep{Twomey1959}, cloud lifetime \citep{Albrecht1989, Small2009}, and cloud morphology \citep{Jiang2009}, but also affect the microscopic properties such as the hygroscopicity (solute effect) and particle growth rate. It is often assumed in many cloud models and microphysics schemes \citep[e.g., ][]{Morrison2005, Thompson2014} that the size of the CCN is relatively small. Therefore, the hygroscopic contribution of CCNs is only considered in the droplet activation process but is omitted in subsequent droplet growth in clouds. Studies such as \citet{Jensen2017} and \citet{blyth2003} found that giant CCN, or GCCN (with a dry radius $> 0.5 \mu m$), mostly observed in marine regions where breaking waves generate giant sea-salt particles, may accelerate condensational growth and act as raindrop embryos.

\subsubsection{Cloud-scale mixing}

As the condensational growth rate of a droplet is mainly associated with its local supersaturation, it has been argued that the fluctuating supersaturation resulting from turbulent fluctuations in the vertical wind can have a large effect on the evolution of the DSD. The first theoretical framework describing this effect was developed by \citet{cooper1989}. He proposed that droplets ending up at the same location within a turbulent cloud can encounter different Lagrangian trajectories and thus experience various condensational growth histories to broaden the DSD. The effect was later coined as "eddy hopping" by \citet{Grabowski2013} to describe the process of a droplet hopping from one large eddy to another and thus experiencing variation of supersaturation along the droplet trajectory. \citet{grabowski2017} demonstrated via a turbulent adiabatic parcel model that turbulence can virtually suppress the DSD narrowing effect during condensational growth. 

\subsection{Entrainment and mixing at the cloud edge}
Another mechanism associated with the effect of mixing on droplet condensational growth is the entrainment near the cloud edge. In the entraining cloud region cloudy and unsaturated air meets and droplets exposed to unsaturated air evaporate completely while other droplets remain unaffected. The result is a reduction in droplet number concentration and fewer droplets to compete for the available water vapor with subsequent faster condensational growth. This mixing process is known as inhomogeneous mixing as opposed to homogeneous mixing where all droplets partially evaporate at a similar pace so that the mean-droplet radius is reduced but the droplet number concentration remains constant. In addition, the fresh CCN entrained into the cloud lead to secondary activation and further broadening of the droplet distribution towards smaller sizes. Nevertheless, numerical studies show that the formation of raindrops by condensational growth still requires high liquid water content (LWC) after droplets experience entrainment \citep{Twomey1966}. Moreover, evidence is lacking to demonstrate that large droplets resulting from entrainment can be reintroduced into high LWC regions \citep{cooper2013}.


In passing we emphasize that the candidate mechanisms mentioned above are not well-quantified and are difficult to be implemented in microphysics parameterizations. Consequently, inaccurate microphysics parameterizations contribute to a large extent to the poor simulation of cloud properties in various models (e.g., LES, CRM, NWP and climate models) \citep{Fan2016}. Untangling cloud microphysical processes and their interaction with cloud dynamics would go a long way toward improving cloud microphysics parameterizations, prediction of cloud cover and rainfall intensity, and better understanding of cloud feedback on climate. This dissertation aims to contribute towards these goals by scrutinizing the effect of turbulence on microphysics and raindrop formation in warm clouds. 
%The geometric collision kernel is a measure of the rate of collision between two droplet sizes under no hydrodynamic effect. It is the average volume swept out by a collector droplet (the bigger one in the droplet pair) per unit time. Within that volume, collisions are expected to happen.



\section{Outstanding questions in warm cloud microphysics}


The representation of cloud processes, especially in shallow clouds, remains a dominant source of uncertainty in climate models \citep[chapter 7]{ipcc2014}. Due to the coarse resolution in current weather and climate models, cloud microphysical processes involved in the formation and evolution of cloud droplets and raindrops, such as condensation, evaporation, collision, coalescence, and breakup, have to be parameterized. The parameterizations provide the thermodynamic and dynamic feedbacks through latent heating/cooling and mass loading, determine the type and intensity of precipitation at the surface, affect the optical properties of clouds which are crucial to atmospheric radiation, and interact with the chemical and aerosol processes. In cloud models such as LES and cloud system resolving models (CRMs), the microphysical representation is not accurate due to poor understanding of the underlying physical processes. Consequentially, there is no benchmark to gauge the parameterization, leading to poor simulation of cloud properties and poor forecasts of precipitation \citep{Fan2016}. In addition, there is no convergence of model results using different schemes. For example, \citet{White2017} found that the uncertainty caused by using different double-moment bulk schemes under the same model framework even exceeds the aerosol effects due to a poor representation of cloud processes (e.g., autoconversion, saturation adjustment and partitioning of hydrometers). Even though bin schemes are more accurate than bulk schemes (bin models calculate the microphysical quantities of each particle bin for every hydrometer category instead of only computing the momentum of a presumed particle size distribution in bulk models), different bin microphysics schemes still yield a wide spread of dynamic and thermodynamic structures of a squall line, qualitatively similar to the spread produced by bulk schemes \citep{Xue2017}. For models such as the CRM or LES with resolution between 10m-100km, microphysics are parameterized in resolved clouds. In models with coarser grids (> 100 km) such as general circulation models (GCMs), microphysics is parameterized within the parameterized clouds. As such, the uncertainty of microphysics schemes is magnified. Therefore, improving the understanding of cloud microphysical process and their interactions with the environment is urgently needed to improve cloud microphysics parameterizations, to provide better constraints on weather and climate models, and to eventually reduce model uncertainty in cloud processes. 

Specifically, this dissertation is focused on investigating and quantifying the effect of turbulence on droplet growth inside the adiabatic cloud core, in the hope of seeking a more accurate parameterization for describing the warm-rain initiation process. The adiabatic core is a region of a cloud where no entrainment of dry air is found. It is argued that the first raindrops are formed here owing to its higher LWC and more large droplets than the surrounding diluted cloud body \citep{Vaillancourt2001, Khain2013}.  The research questions to be answered by this study are listed below:


1) What is the role of turbulence in droplet geometric collision (i.e., the collision that does not consider hydrodynamic droplet interactions)?  What is the relative importance of different scales of turbulent motion on droplet collisions? Can we quantify the turbulent effects and find an accurate parameterization of the collision rate?

2) How does turbulence impact the droplet hydrodynamic interaction and thus modify the collision efficiency? 

3) How does the condensational process interact with the collisional process in a turbulent environment? How does turbulence modulate such interactions?

4) Finally, how does the droplet size distribution evolve under different turbulent conditions? And how does turbulence modulate the rain formation time?


\section{Methodology}

\subsection{Direct numerical simulation}

To tackle the research questions raised in the previous section, a direct numerical simulation (DNS) approach is employed to simulate the cloud microphysical processes in the turbulent, adiabatic region inside a cumulus cloud, aka. the adiabatic core. In each chapter, the DNS code is modified in such a way that the specific physical processes are resolved to scrutinize the turbulence effect on those processes. The model consists of two components: dynamics (i.e., turbulent flow) and cloud microphysics (i.e., individual droplet Lagrangian movement, detection of droplet collisions, and/or droplet diffusional growth). 

Overall, the DNS aims to simulate explicitly continuous droplet growth in a turbulent, supersaturated environment inside the cloud. The whole computational domain is regarded as an air parcel ascending from the cloud base. The turbulence is generated by solving the incompressible Navier-Stokes equations. The disturbance flow induced by flow passing over the droplet surface is solved and super-imposed onto the turbulent flow such that all droplet-collision-relevant dynamics are appropriately included. For droplets, they are traced in a Lagrangian framework, and their motion is governed by the drag of the flow and the effect of sedimentation (i.e., gravity). To collect the turbulent collision statistics of designated droplet sizes, droplets are not allowed to merge by collision-coalescence (Chapter \ref{sec:ch2} and first half of Chapter \ref{sec:ch3}). In this circumstance, once two droplets collide, they are removed immediately from the original location and replaced randomly in space. To simulate droplet collisional growth (second half of Chapter \ref{sec:ch3} and Chapter \ref{sec:ch4}), a “one-step approach” is used. Specifically, once two droplets collide, they are allowed to merge to become one large droplet. This direct approach provides the most accurate and realistic way to study warm-rain initiation. To include condensation, the temperature and mixing ratio fields are coupled to the turbulence and used to calculate the supersaturation fluctuation in Chapter \ref{sec:ch4}. With the presence of the microscopic dynamical and thermodynamic fields, droplet growth by condensation and collision-coalescence are simultaneously resolved. The specific modification and validation are described in detail in each chapter.

\subsection{Parallelization}
To improve the performance of the model to allow simulations with a sufficiently large domain size (i.e., a large number of grid points), the model is parallelized using the message passing interface (MPI) technique. The domain is divided into equal vertical slabs (i.e. 1D decomposition), each of which is assigned to one processor to calculate the dynamics and microphysics within that slab. At each time step the location of every droplet is updated and checked to see if the new location lies beyond the slab boundaries. If so, the droplet will be transferred to the processor responsible for the slab that the droplet moves into. 
%In addition, to avoid droplets moving more than a complete slab width within a timestep, slabs are set to be no wider than $2\Delta x$.



\section{Dissertation outline}

The structure of the remaining portions of the dissertation are as follows:

\subsubsection{Chapter \ref{sec:ch2}: }
Chapter \ref{sec:ch2} investigates and quantifies the turbulence effect of droplet geometric collisions by DNS. At this stage, we do not include droplet hydrodynamic interaction so that all collisions considered are geometric. I.e., the collisions are a result of the geometric overlap of two droplets that are on collision course unaffected by the disturbance flow around the droplet. In addition, we do not allow droplets to grow and only collect collision statistics at different turbulence intensities. A sensitivity study is conducted over a broad range of computational Reynolds numbers to determine the important scales pertaining to droplet collisions in cumulus clouds. Following the statistical analysis, a new parameterization of the turbulent geometric collision kernel is proposed. Different from previous parameterizations, our scheme is able to describe both same-sized collisions and cross-sized collisions simultaneously, instead of having to represent the situations separately. The new scheme also does not depend on the Reynolds number, which is usually based on the computational domain size, but is solely determined by the turbulent dissipation rate and the droplet size. Compared to other formulations, our formula performed well in predicting droplet collision rates in stratiform and cumulus clouds (i.e. dissipation rate less than 600 $cm^2s^{-3}$).
\subsubsection{Chapter \ref{sec:ch3}: }
Chapter \ref{sec:ch3} scrutinizes the turbulence effect of the droplet collision efficiency and hydrodynamic collisions. We incorporate the droplet disturbance flow field into the simulated turbulent flow and hence hydrodynamic interactions between droplets at a close separation distance are explicitly resolved. The first half of Chapter \ref{sec:ch3} examines how turbulence modulates droplet hydrodynamic interactions to modify the collision efficiency between cloud droplets of various sizes. In the second half of Chapter \ref{sec:ch3}, we conduct DNS experiments of droplet growth by hydrodynamic collisions to illustrate the evolution of droplet size distribution in various turbulent environments.
\subsubsection{Chapter \ref{sec:ch4}: }
Chapter \ref{sec:ch4} extends the study of the previous chapters and is focused on understanding the impact of turbulence on continuous droplet growth in shallow clouds. By including the droplet condensational growth process in the DNS model, droplets are allowed to grow by collision-coalescence and condensation at the same time. Numerical experiments on collision-only, condensation-only, and collision-condensation are conducted and compared to investigate the contribution of each process and the interactions of both processes to raindrop formation. Simulations of different turbulent intensities are conducted to investigate the impact of turbulence on those processes. 

\subsubsection{Chapter \ref{sec:ch5}:}
Chapter \ref{sec:ch5} summarizes the dissertation's main contributions and discusses the limitation of this work and the future research needed in this area.
